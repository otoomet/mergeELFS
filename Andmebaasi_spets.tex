% utf-8
\documentclass[a4paper]{article}
\usepackage{amsmath}
\usepackage[T1]{fontenc}
\usepackage[utf8]{inputenc}
\usepackage[estonian, english]{babel}
\usepackage{paralist}
\usepackage{xspace}
\usepackage[estonian]{babel}

\newcommand{\FALSE}{\textbf{FALSE}\xspace}
\newcommand{\NA}{\textbf{NA}\xspace}
\newcommand{\R}{\textsf{R}\xspace}
\newcommand{\TRUE}{\textbf{TRUE}\xspace}

\title{Estonian LFS panel}
\author{Indrek Seppo \and Ott Toomet}

\begin{document}
\maketitle

\selectlanguage{estonian}
\section{Andmebaas}
ETU aegridade andmebaasi peamine ülesanne on kogude ETU-dest kõik
olemasolevad palgavaatlused ja võimaldada neid ühtsel viisil kasutada.
Põhirõhk on muutujatel mida on võimalik jälgida üle kõigi aastate.

Andmebaas koosneb kolmest komponendist
\begin{itemize}
\item Andmebaas ise.  Andmebaas on ASCIIpõhine tsv-faili kujul.
\item Programm, mis selle algandmetest genereerib.  
  \selectlanguage{english}  The program contains two main files: one
  for transforming the old structure (1995-2000) and the other for
  transforming the new structure (2000-).
\item Documentation (this text).
\end{itemize}

Our aim is to represent numerically only real numeric data (such data
where the common mathematical operations like addition and division
make sense).  Other numerical codes (e.g. counties, ownership codes
etc) will be represented as categorical variables (factors) where the
possibly numerical codes are in labels.  \selectlanguage{estonian}


\paragraph{Struktuur}

\subparagraph{Küsitlust, vaatlust ning indiviidi identifitseerivad
  abiandmed}
Andmebaas koosneb vaatlustest, iga rida vastab ühele vaatlusele, ehk
esitab ühe indiviidi andmed ühel konkreetsel ajahetkel. Ühe indiviidi
kohta võib olla mitu vaatlust: 
\begin{enumerate}
\item kuni aastani 2000 küsitleti andmeid ka retrospektiivselt, st indiviidi
  kohta on vaatlus nii küsitluse hetkele lähedase ajahetke kohta kui varasemate
  ajahetkede kohta. Seega vastab ühele küsitlusele mitu vaatlust sama indiviidi
  kohta.
  \selectlanguage{english}
\item Since the 2000 Q3, the same individuals are surveyed in several
  waves while retrospective part is dropped.  Hence the database
  includes several lines of data for every individual, describing
  several survey weeks.

  Note that there are several inconsistencies across the waves.  A
  number of variables (such as ethnicity) may not be well defined.
  This may also be because of errors in identifiers (such as swapping
  around household member, which leads to inconsistent gender and year of
  birth).  No attempt is made to correct for these problems.
\end{enumerate}

The first variables in the data indentify the survey wave (where the
data originate from), individual, retrospectivity, and date for the
observation.  Note that a single individual may have several
observations for different dates.  The most important variables are
perhaps the individual indentifier \verb|idPerson| and date of the
observation \verb|date|.
\begin{description}
\item[id] The unique code of the respondent/survey.  This is in the
  form YYYYQHHHHHHMMYYQ, where YYYY is the year and Q the quarter of
  the survey, HHHHHH is the household id and MM is the household
  member id.  Note that the retrospective and current data share the
  same \texttt{id} although the make different lines in the database.
  Use line number if you need unique observation code.
\item[idHousehold] Unique code for the household in the form
  YYYYQHHHHHH.  The YYYY and Q are the year and quarter, household
  entered the survey for the first time (\texttt{wave} = 1).

  Note that some of the waves are missing for some households.  We
  still base the id on the year and quarter following \texttt{wave}.

  Note also that household codes HHHHHH are recycled between waves.
  HHHHH does not form unique household code across waves!
  \selectlanguage{estonian}
\item[indiviidi kood, \textit{idPerson}] Indiviidi kood
  identifitseerib konkreetse indiviidi, kui tema kohta esineb mitmeid
  vaatlusi (nagu toimub alates 2000-ndast aastast). 1995-1999 aasta ETUdes on
  indiviidi kood sama, mis vaatluse kood. Edaspidi on see kujul HHHHHHMM000, kus
  HHHHHH tähistab leibkonnanumbrit ja MM leibkonnaliikme numbrit (lisatud 000 on
  jäetud tulevaste võimalike uuringulainete eristamiste tarvis). Iga
  leibkonnaliige osaleb uuringus kuni neli korda, kahel järjestikusel kvartalil
  ning peale kahe kvartalist vaheaega uuesti kahel järjestikusel kvartalil.
\item[retrospektiivus, \textit{retrospective}] võimaldab eristada
  retrospektiivseid ja oleviku (õieti lähimineviku - 1995 aasta alguses küsitud
  1994 aasta sügise palk on märgitud mitteretrospektiivseks) kohta käivaid
  vaatlusi. Retrospektiivsus on esitatud loogilise tunnusena kujul TRUE/FALSE, kus
  TRUE tähendab, et vaatlus on retrospektiivne, ning FALSE, et
  vaatlus käib lähimineviku kohta. Retrospektiivsete vaatluste korral tuleb
  arvestada nii meenutamisega seotud vigadega kui valimi muutumisega (1995 aastal
  uuritud 16-75 aasta vanused olid 1989 9-69 aasta vanused), mistõttu ei pruugi
  aastate kaupa tehtavad võrdlused (näiteks 1989 ja 1993 a keskmine palk) olla
  adekvaatsed, küll aga annavad nad teatud sissevaate inimese käitumise
  dünaamikasse.
\item[kuupäev, \textit{date}] ajahetk, mille kohta vaatlus käib. Ajahetk on
  antud POSIX standardi järgi kuupäeva ja kellaajase täpsusega kujul YYYY-MM-DD
  HH:MM:SS, kuigi küsimus võib olla püstitatud kujul "Kui palju te teenisite 1989
  a. sügisel?". Sellisel juhul on kuupäevaks märgitud perioodi keskmine
  (konkreetse näite korral 1. oktoober 1989), kellaajaks keskpäev.
\end{description}
\selectlanguage{english}
Note that \emph{idPerson} and \emph{date} do not always form unique
couples.  In some rare cases (2000-Q1 wave) the survey is conducted on
the date which corresponds to a retrospective observation.  The user
is responsible for identifying and correcting for these problems.
\selectlanguage{estonian}

\subsection{Küsitletava üldandmed}
Küsitletava üldandmete blokis on küsitletava sugu, rahvus, sünniaasta,
perekonnaseis jmt.

\selectlanguage{english}
\begin{description}
\item[sex] 1 -- male; 2 -- female
\item[birthYear] year of birth, 4-digits 
\item[age] age of the respondent in years. As the birthday information
  in (mostly) missing, it may be wrong by up to 6 months.  

  The age is assumed to be >= 0.  This may be a problem in case of
  data entry typos.

  In case of wrong or missing survey date, the survey date is assumed
  to be a specific date during the survey wave.  It may have an impact
  on age as well, as $\text{age} = \text{survey date} -
  \text{(assumed) birthday}$. 

\item[nonEst] ethnic background other than Estonian.  

  \selectlanguage{estonian}
\item[Eestisse immigreerumise aeg, \textit{immigrTime}] aasta, millal tuli
  Eestisse elama (kujul YYYY)

  \selectlanguage{english}
\item[maritalStatus]
  \begin{inparaitem}
  \item[1)] single;
  \item[2)] co-habiting;
  \item[3)] married;
  \item[4)] widow;
  \item[5)] divorced;
  \item[6)] separated.
  \end{inparaitem}
  \begin{itemize}
  \item The information is based on information on survey week, the
    retrospective part may be wrong.
  \item The coding was changed in 2008 and hence the data from that
    date on, is based on different variables/questions.
  \end{itemize}

\item[InterEthHH] Interethnic household.  Does the household include
  ethnic Estonians and Non-Estonians?  The other ethnic groups are not
  distinguished. 

  The information is based on the survey week and extrapolated to the
  retrospective part.  

  Data for LFS 1995 is missing, and all the households are coded as
  mono-ethnic.  


  \subsubsection{Language skills}
  \label{sec:language}

  ELFS includes self-reported language skills data.  Note the related
  caveats: 
  \begin{itemize}
  \item The data is collected for the survey week only.  However, the
    program writes the same skill levels for the retrospective
    observations as well.
  \item 1998-2007, the domestic language data only includes
    information whether Estonian and/or other languages are spoken at
    home.  The program assumes 'other' means Russian (which is true in
    most cases, look data for 1995 and 1997).  From 2008 on, the
    information is present for 'Estonian', 'Russian' and 'other'
    domestic languages.
  \end{itemize}

\item[estLevel] Estonian language skills:
  \begin{inparaitem}
  \item[blank] no knowledge.
  \item['3'] understanding;
  \item['2'] speaking;
  \item['1'] speaking and writing;
  \item['home'] one of domestic languages;
  \end{inparaitem}
  The counterintuitive order retains the values in the original survey
  data.  However, the \R factor is ordered (increasingly) according to
  the skill level.
\item[rusLevel] Russian skills;
\item[engLevel] English skills;
\item[FILevel] Finnish skills;

  \subsubsection{Geography}
  \label{sec:geography}

\item[residenceCounty]
  \label{item:residenceCounty} county of residence, code of Estonian
  county.  \NA if not Estonian resident.
  \begin{inparaitem}
  \item[37] Harju (including Tallinn);
  \item[39] Hiiu;
  \item[44] Ida-Viru;
  \item[49] Jõgeva;
  \item[51] Järva;
  \item[57] Lääne;
  \item[59] Lääne-Viru;
  \item[65] Põlva;
  \item[67] Pärnu;
  \item[70] Rapla;
  \item[74] Saaremaa;
  \item[78] Tartu;
  \item[82] Valga;
  \item[84] Viljandi;
  \item[86] Võru.
  \end{inparaitem}
\item[residenceMunicipality] municipality of residence.  Corresponds
  mostly to EHAK2008 specification.  However, certain territorial
  units (as Tallinn and Kohtla-Järve neighbourhoods) are merged.
\end{description}

\subsection{Education}
\begin{description}
\item[edu] current highest completed education.  Three levels:
  \emph{<=basic}, \emph{highSchool}, \emph{college}.
\item[isced97] 1-digit ISCED97 for the highest completed education.
  Only available for the survey week.  Note: for 1995, 1997, 2011, and
  2012 waves this involves a fair amount of guesswork.  The variable
  is ready coded for the intermittent years.
\item[studying] whether currently studying (1) or not (0).  This is
  about studying at \emph{school}, not about training courses, courses
  for unemployed and hobby courses.
\end{description}


\subparagraph{Household Data}

Note: the data includes the \emph{household} members, we do not
analyze the eventual family relations between the members.  In
particular, children are not necessarily kids of the responent(s).
\selectlanguage{estonian}
Lubatud on ajutine eemalolek, mis
siinses kontekstis tähendab ilmselt õpinguid või haiglasviibimist.
Retrospektiivsete vastuste juures on arvestatud üksnes küsitluse hetkel
leibkonnas viibivaid lapsi (kelle vanusest on lahutatud küsitluse hetke ning
vaatluse hetke vaheline aeg), mille tõttu kannatab olulisult täpsus! See
tähendab, et arvestamata on vahepeal surnud lapsed ning valesti võivad olla
arvestatud leibkonda vahetanud lapsed. Eriti oluliseks muutub see 1995 aasta ETU
retrospektiivse osa kohta, mis ulatub kuue aasta taha, ja kus teoreetiliselt
võib ilmneda, et küsitluse hetkel leibkonna moodustanud 15-aastased loetakse
kuus aastat varem samuti iseseisvaks leibkonnaks, kus nad esinevad alla
kümneaastaste lastena. 
Tähelepanu tuleb pöörata ka sellele, et andmetes on pereliikmete sünniajad
esitatud kuu täpsusega, sellisel juhul on arvestuslikuks kuupäevaks loetud 14
ehk arvutuslikult loetakse eelneva aasta oktoobrikuus sündinud oktoobrikuu
uuringu ajaks alati üheaastaseks jne.
\begin{description}
\item[kids0.3] Leibkonnas uuritaval hetkel olnud 0-3 aastaste laste
  arv, retrospektiivsetel perioodidel ebatäpne.
\item[kids4.6] Leibkonnas uuritaval hetkel olnud 4-6 aastaste laste
  arv, retrospektiivsetel perioodidel ebatäpne.
\item[kids7.17] Leibkonnas uuritaval hetkel
  olnud 7-17 aastaste laste arv, retrospektiivsetel perioodidel
  ebatäpne.
  \selectlanguage{english}
\item[age63.64] \# of household members in age interval $[64,65)$.
\item[age65.] \dots in interval $[65, \infty)$.
\end{description}

\subsection{Workplace Data}

\begin{description}
\item[industry] industry:
  \begin{inparaitem}
  \item[A] Agriculture, hunting and forestry;
  \item[B] Fishing;
  \item[C] Mining and quarrying;
  \item[D] Manufacturing;
  \item[E] Electricity, gas and water supply;
  \item[F] Construction;
  \item[G] Wholesale and retail trade; repair of motor vehicles etc.;
  \item[H] Hotels and restaurants;
  \item[I] Transport, storage and communication;
  \item[J] Financial intermediation;
  \item[K] Real estate, renting and business activities;
  \item[L] Public administration and defence; compulsory social security;
  \item[M] Education;
  \item[N] Health and social work;
  \item[O] Other community, social and personal service activities;
  \item[P] Activities of households;
  \item[Q] Extra-territorial organizations and bodies.
  \end{inparaitem}
\item[nWorkers] number of employees at the firm.  Following
  classes:
  \begin{inparaenum}
  \item 1-10;\footnote{1-9 for ELFS 1995}
  \item 11-19;\footnote{10-19 for ELFS 1995}
  \item 20-49;
  \item 50-99;
  \item 100-199;
  \item 200-499;
  \item 500-999;
  \item 1000-;
  \end{inparaenum}
\item[ownership] owner of the firm:
  \begin{itemize}
  \item[10] government (Estonian State)
  \item[20] local government
  \item[30] private, Estonian
  \item[40] private, foreign
  \item[50] private, mixed Estonian and foreign
  \item[90] other
  \end{itemize}
\item[publicSector] (logical) the establishment is owned by public sector, and
  is not operating on the free market.

  \NA, if either this information, or information about the owners is missing.
\item[workCountry] country of the workplace.  Until 1997 the
  individual countries are present, later coded as '1' = Estonia; '2'
  = another country.  Individual countries:
  \begin{inparaitem}
    \item[4] Afganistan;
    \item[8] Albaania;
    \item[10] Antarktika;
    \item[12] Alzeeria;
    \item[16] Ameerika Samoa (Ida-Samoa);
    \item[20] Andorra;
    \item[24] Angola;
    \item[28] Antigua ja Barbuda;
    \item[31] Aserbaidz'an;
    \item[32] Argentina;
    \item[36] Austraalia;
    \item[40] Austria;
    \item[44] Bahama Saared;
    \item[48] Bahrein;
    \item[50] Bangladesh;
    \item[51] Armeenia;
    \item[52] Barbados;
    \item[56] Belgia;
    \item[60] Bermuda;
    \item[64] Bhutan;
    \item[68] Boliivia;
    \item[70] Bosnia ja Hertsegoviina;
    \item[72] Botswana;
    \item[74] Bouvet Saar;
    \item[76] Brasiilia;
    \item[84] Belize;
    \item[86] Briti India ookeani ala;
    \item[90] Solomoni Saared (Briti);
    \item[92] Neitsisaared (Briti);
    \item[96] Brunei;
    \item[100] Bulgaaria;
    \item[104] Myanmar (birma);
    \item[108] Burundi;
    \item[112] Valgevene;
    \item[116] Kambodz'a;
    \item[120] Kamerun;
    \item[124] Kanada;
    \item[132] Cabo Verde (Roheneemesaared);
    \item[136] Kaimanisaared;
    \item[140] Kesk-Aafrika Vabariik;
    \item[144] Sri Lanka;
    \item[148] Ts'aad;
    \item[152] Chile;
    \item[156] China;
    \item[158] Taiwan;
    \item[162] Jõulusaar;
    \item[166] Kookosesaared;
    \item[170] Kolumbia;
    \item[174] Komoorid;
    \item[175] Mayotte;
    \item[178] Kongo;
    \item[180] Sair;
    \item[184] Cooki saared;
    \item[188] Costa Rica;
    \item[191] Horvaatia;
    \item[192] Kuuba;
    \item[196] Küpros;
    \item[203] Ts'ehhimaa;
    \item[204] Benin;
    \item[208] Taani;
    \item[212] Dominica;
    \item[214] Dominikaani Vabariik;
    \item[218] Ecuador;
    \item[222] Salvador;
    \item[226] Ekvatoriaal-Guinea;
    \item[231] Etioopia;
    \item[232] Eritrea;
    \item[233] Eesti;
    \item[234] Fääri saared;
    \item[238] Falklandi (Malviini) saared;
    \item[239] Lõuna-Georgia ja Lõuna-Sandwich;
    \item[242] Fidz'i;
    \item[246] Soome;
    \item[249] Prantsuse emamaa;
    \item[250] Prantsusmaa;
    \item[254] Prantsuse Guajaana;
    \item[258] Prantsuse Polüneesia;
    \item[260] Prantsuse Lõunaalad;
    \item[262] Djibouti;
    \item[266] Gabon;
    \item[268] Gruusia;
    \item[270] Gambia;
    \item[276] Saksamaa;
    \item[288] Ghana;
    \item[292] Gibraltar;
    \item[296] Kiribati;
    \item[300] Kreeka;
    \item[304] Gröönimaa;
    \item[308] Grenada;
    \item[312] Guadeloupe;
    \item[316] Guam;
    \item[320] Guatemala;
    \item[324] Guinea;
    \item[328] Guyana;
    \item[332] Haiti;
    \item[334] Heard ja McDonald;
    \item[336] Vatikan;
    \item[340] Honduras;
    \item[344] Hongkong;
    \item[348] Ungari;
    \item[352] Island;
    \item[356] India;
    \item[360] Indoneesia;
    \item[364] Iraan;
    \item[368] Iraak;
    \item[372] Iirimaa;
    \item[376] Iisrael;
    \item[380] Itaalia;
    \item[384] Elevandiluurannik;
    \item[388] Jamaica;
    \item[392] Jaapan;
    \item[398] Kasahstan;
    \item[400] Jordaania;
    \item[404] Keenia;
    \item[408] Korea RDV;
    \item[410] Korea Vabariik;
    \item[414] Kuveit;
    \item[417] Kõrgõzstan;
    \item[418] Laos;
    \item[422] Liibanon;
    \item[426] Lesotho;
    \item[428] Läti;
    \item[430] Libeeria;
    \item[434] Liibüa;
    \item[438] Liechtenstein;
    \item[440] Leedu;
    \item[442] Luksemburg;
    \item[446] Macau;
    \item[450] Malgassi Vabariik (Madagaskar);
    \item[454] Malawi;
    \item[458] Malaisia;
    \item[462] Maldiivid;
    \item[466] Mali;
    \item[470] Malta;
    \item[474] Martinigue;
    \item[478] Mauritaania;
    \item[480] Mauritius;
    \item[484] Mehhiko;
    \item[492] Monaco;
    \item[496] Mongoolia;
    \item[498] Moldova;
    \item[500] Monsterrat;
    \item[504] Maroko;
    \item[508] Mosambiik;
    \item[512] Omaan;
    \item[516] Namiibia;
    \item[520] Nauru;
    \item[524] Nepal;
    \item[528] Holland;
    \item[530] Hollandi Antillid;
    \item[533] Aruba;
    \item[540] Uus-Caledoonia;
    \item[548] Vanuatu;
    \item[554] Uus-Meremaa;
    \item[558] Nicaragua;
    \item[562] Niger;
    \item[566] Nigeeria;
    \item[570] Niue;
    \item[574] Norfolk;
    \item[578] Norra;
    \item[580] Põhja-Mariani Saared;
    \item[581] USA väikesed üksiksaared;
    \item[583] Mikroneesia;
    \item[584] Marshalli saared;
    \item[585] Palau;
    \item[586] Pakistan;
    \item[591] Panama;
    \item[598] Paapua Uus-Guinea;
    \item[600] Paraguay;
    \item[604] Peruu;
    \item[608] Filipiinid;
    \item[612] Pitcairn;
    \item[616] Poola;
    \item[620] Portugal;
    \item[624] Guinea-Bissau;
    \item[626] Ida-Timor;
    \item[630] Puerto Rico;
    \item[634] Qatar;
    \item[638] Reunion;
    \item[642] Rumeenia;
    \item[643] Venemaa;
    \item[646] Ruanda;
    \item[654] Saint Helena;
    \item[659] Saint Kitts ja Nevis;
    \item[660] Anguilla;
    \item[662] Saint Lucia;
    \item[666] Saint Pinrre ja Miguelon;
    \item[670] Saint Vincent ja Grenadiinid;
    \item[674] San Marino;
    \item[678] Sao Tome ja Principe;
    \item[682] Saudi Araabia;
    \item[686] Senegal;
    \item[690] Seisellid;
    \item[694] Sierra Leone;
    \item[702] Singapur;
    \item[703] Slovakkia;
    \item[704] Vietnam;
    \item[705] Sloveenia;
    \item[706] Somaalia;
    \item[707] Serbia;
    \item[710] Lõuna-Aafrika Vabariik;
    \item[716] Zimbabwe;
    \item[724] Hispaania;
    \item[732] Lääne-Sahara;
    \item[736] Sudaan;
    \item[740] Surinam;
    \item[744] Teravmäed [Svalbard] ja Jan Mayen;
    \item[748] Svaasimaa;
    \item[752] Rootsi;
    \item[756] S'veits;
    \item[760] Süüria;
    \item[762] Tadz'ikistan;
    \item[764] Tai;
    \item[768] Togo;
    \item[772] Tokelau;
    \item[776] Tonga;
    \item[780] Trinidad ja Tobago;
    \item[784] Araabia Ühendemiraadid;
    \item[788] Tuneesia;
    \item[792] Türgi;
    \item[795] Turkmeenia;
    \item[796] Turks ja Caicos;
    \item[798] Tuvalu;
    \item[800] Uganda;
    \item[804] Ukraina;
    \item[807] Makedoonia;
    \item[818] Egiptus;
    \item[826] Suurbritannia (Ühendkuningriik);
    \item[834] Tansaania;
    \item[840] USA
    \item[850] USA Neitsisaared;
    \item[854] Burkina Faso (end.Ülem-Volta);
    \item[858] Uruguay;
    \item[860] Uzbekistan;
    \item[862] Venezuela;
    \item[876] Wallis ja Futuna;
    \item[882] Lääne-Samoa;
    \item[887] Jeemen;
    \item[891] Jugoslaavia;
    \item[894] Sambia;
  \end{inparaitem}
\item[workCounty] county of the firm or establishment.  \NA if not
  working or not working in a defined geographical location.  Codes:
  see \emph{residenceCounty}, page~\pageref{item:residenceCounty}.
\item[workMuni] municipality of work.  Corresponds mostly to EHAK2008
  specification.  However, certain territorial units (as Tallinn and
  Kohtla-Järve neighbourhoods) are merged.  
\item[employeeStatus] employer/employee type:
  \begin{inparaitem}
    \selectlanguage{estonian}
  \item[1)] palgatöötaja;
  \item[2)] palgatöötaja(te)ga ettevõtja;
  \item[3)] palgatööjõuga talupidaja;
  \item[4)] üksikettevõtja;
  \item[5)] palgatööjõuta talupidaja;
  \item[6)] vabakutseline;
  \item[7)] palgata töötaja pereettevõttes, talus;
  \item[8)] tulundusühistu liige;
  \item[10)] ettevõtja palgatöötaja(te)ta;
  \item[11)] other.
  \end{inparaitem}
\item[töökogemus ettevõttes] \textit{experienceInCompany}] mitu aastat on selles
ettevõttes töötanud
\end{description}


\selectlanguage{english}
\subsection{Work-related background data}

\begin{description}
\item[sidejob] respondent had more jobs beside of the main (first)
  job.  Codes:
  \begin{inparaitem}
  \item[1] yes;
  \item[2] from time to time (only present for the retrospective part);
  \item[3] no;
  \item[\NA] not working.    
  \end{inparaitem}
  Note that the information differs for survey week and the
  retrospective part.

  The retrospective information is based on the presence of second job
  occupation, except for ELFS 1995 where the direct question is present.
\item[unionMember] membership of union.  Data only present for the
  survey week since 1997.  Missing for 1995.  

  One can trust the answer \TRUE.  However, if the individual is
  not working; or there is no union at the workplace, the answer is
  \NA.  In 2007, however, it is \FALSE in the latter case.
\item[activityStatus] Answer to the question ``what is your main
  activity status'' (in 2001 wave):
  \begin{inparaenum}
  \item employed;
  \item employed student;
  \item employed retired person;
  \item employed disabled person;
  \item unemployed;
  \item student;
  \item retired;
  \item disabled;
  \item on parental leave;
  \item military service;
  \item imprisoned;
  \item not working for other reasons;
  \item other
  \end{inparaenum}
  See also: \emph{workForceStatus}
\item[workForceStatus] Main labour force status:
  % based on 'status', documented in ETU2002_kirjeldus.sxc
  \begin{inparaenum}
  \item employed;
  \item unemployed;
  \item inactive;
  \end{inparaenum}
  It is based on the variable \emph{status} for most of the waves.
  However, for 1997 survey week, it is composed on questions about
  job, employer status, farmer status and temporary holiday.

  For retrospective observations, it is based on the employment and
  unemployment spells.
\selectlanguage{estonian}
\item[Töökogemus] \textit{firstJob}] Aasta ja kuu, mil asus esmakordselt tööle.
\item[tööturu] seisund, \textit{activityStatus}] kas töötab, on töötu või
inaktiivne
\selectlanguage{english}
\item[wage] salary at the main job.

  The income from secondary jobs is not included.

  Gross salary until ELFS 1997, thereafter net salary.

  For retrospective observations, it is based on questions: ``what was
  your salary at that month''.  For the survey week, the question is
  ``what was your salary at this job last month''.

  Note that in many cases salary = 0.
\end{description}

\subsection{Job-related data}
\begin{description}
\item[occupation] Occupation:
  \begin{inparaitem}
  \item[0] Armed forces;
  \item[1] Legislators, senior officials and managers;
  \item[2] Professionals;
  \item[3] Technicians and associate professionals;
  \item[4] Clerks;
  \item[5] Service workers and shop and market sales workers;
  \item[6] Skilled agricultural and fishery workers;
  \item[7] Craft and related trade workers;
  \item[8] Plant and machine operators and assemblers;
  \item[9] Elementary occupations;
  \end{inparaitem}
\item[howFoundJob] How the job was found.  Present for all the jobs
  until 2000 Q2, since 2000 Q3 only for the main job at the survey
  week.  Codes: 
  \begin{inparaenum}
  \item[0] was appointed after graduation;
  \item[1] Asked relatives or friends;
  \item[2] Responded to job advertisements;
  \item[3] Placed job seeking advertisements;
  \item[4] Contacted employers directly;
  \item[5] Sought job through the State Employment Bureau (SEB) [RECORD NAME];
  \item[6] Sought job through private employment bureau [RECORD NAME];
  \item[7] By starting own business/farm;
  \item[8] Started working on a family farm / in a family enterprise;
  \item[9] Returned to my previous job after military service;
  \item[10] Returned to my previous job after parental leave;
  \item[11] In connection with closing down / reorganisation of the enterprise;
  \item[12] (Better) job was offered, did not seek it;
  \item[13] Second job became main job;
  \item[14] OTHER [RECORD];
  \end{inparaenum}
\item[workhoursTotal] Total working hours.  

  Only present for survey weeks for waves 1995--2000Q2
\item[partOrFullTime] Whether working part-time (2) or full-time job
  (1).  For retrospective observations this is based on question:
  ``did you have full time or part time job''.  For the survey week,
  the question is based on usual working hours at the first job;
  part-time job means working less than 35 hours.  Note that periods
  of part-time work are not accounted for.
  \begin{inparaitem}
  \item[1] full time;
  \item[2] part time;
  \item[3] alternately full- and part time (pre 2000/3 retrospective
    observations only).
  \end{inparaitem}
\item[partTimeHours] Working hours, only correct for part-time workers
  (\emph{partOrFullTime} = 2).
\item[partTimeReason] Why the individual is working part-time.  The
  answers are recoded from the original.  Note that slightly different
  information is available in different waves and for
  survey/retrospective part.
  \begin{inparaenum}
  \item[1] Studies
  \item[2] Illness or injury
  \selectlanguage{estonian} 
  \item[3] Rasedus- või sünnituspuhkus
  \item[4] Lapsehoolduspuhkus
  \item[5] Vajadus hoolitseda laste eest
  \item[6] Vajadus hoolitseda laste või hooldamist vajavate täiskasvanute eest
  \item[7] Muud isiklikud või perekondlikud põhjused
  \item[8] Vajadus hoolitseda teiste pereliikmete eest
  \item[9] Tellimuste või töö vähesus
  \item[10] Remont, tehniline rike vms
  \item[11] Materjali- või tooraine vähesus
  \item[12] Minu tööl loetakse täistööajaks vähem kui 35 tunnist töönädalat
  \item[13] Ei ole täisajatööd leidnud
  \item[14] Soovisin säilitada täispensioni
  \item[15] Ei soovinud täisajaga töötada
  \item[16] MUU [KIRJUTAGE TABELISSE
  \end{inparaenum}
  \selectlanguage{english}
\end{description}

\subsection{Unemployment}
\label{sec:unemployment}

\begin{description}
\item[UDuration] If unemployed in the survey week: the unemployment
  duration (up to the survey date) in weeks.

  Note: it is assumed that the individual started the job search at
  1st of the corresponding month.  The start date is based on start of
  jobs search information, not necessarily end of previous job spell,
  i.e. on-the-job search is included.  This leads to a number of
  ridiculously long durations.

  There are a few negative values.  In these cases the reported job
  search starts from the month, following the interview.
\item[UBEligibility] Eligibility for unemployment benefit measures:
  \begin{inparaitem}
  \item[0]: not eligible;
  \item[1]: eligible for Unemployment Insurance Benefits
    (töötuskindlustushüvitis).
  \item[2]: eligible for Unemployment Allowance (töötu toetus)
  \end{inparaitem}
\item[UIBenefits] Size of last month UI benefits (krooni)
\item[UAllowance] Size of last month U Allowance (krooni)
\item[howSearchJob] How the responded searched for a job during the
  past 4 weeks.  16-bit integer with the bits denoting following (from
  the leftmost bit):
  \begin{inparaitem}
  \item[01)] Searched through the public employment agency;
  \item[02)] \qquad through a private employment agency;
  \item[03)] Asked employers directly;
  \item[04)] Asked relatives, friends;
  \item[05)] Answered ``help wanted'' ads;
  \item[06)] Avaldasin ise tööotsimiskuulutusi;
  \item[07)] Jälgisin tööpakkumiskuulutusi;
  \item[08)] Käisin tööandja juures intervjuul, tegin testi või eksami;
  \item[09)] Otsisin maad, ruume, seadmeid, töötajaid jms ettevõtluse alustamiseks / talu rajamiseks;
  \item[10)] Taotlesin ettevõtte/talu registreerimist, tegevusluba, kauplemisluba, laenu;
  \item[11)] Ootasin vastust töösaamise avaldusele;
  \item[12)] Ootasin teadet Tööturuametist;
  \item[13)] Ootasin konkursi tulemusi ametikohale riigi- või omavalitsusasutuse (avaliku] teenistuse seaduse järgi);
  \item[14)] Ootasin varem kokkulepitud töö algust kuni 3 kuu pärast;
  \item[15)] Ootasin varem kokkulepitud töö algust rohkem kui 3 kuu pärast;
  \item[16)] MUU [KIRJUTAGE].
  \end{inparaitem}

  Only present for 2000-Q3 and later.
\item[mainSearchMethod] which of the job search methods (as listed for
  \emph{howSearchJob}) is the most important one.
\item[reservationWage] Reservation wage, gross (bruto)\footnote{Note
    that the employer has to pay a number of contributions on top of
    the \emph{bruto}-wage.}, kroons per month.
\item[reservationWageLevels] Reservation wage, in intervals.
  Intervals are year-dependent: for years 2000-2007 they are:
  \begin{inparaitem}
  \item[1)] At least 500 kroons a month;
  \item[2)] At least 1,000 kroons a month;
  \item[3)] At least 2,000 kroons a month;
  \item[4)] At least 3,000 kroons a month;
  \item[5)] At least 4,000 kroons a month;
  \item[6)] At least 5,000 kroons a month;
  \item[7)] At least 10,000 kroons a month;
  \item[8)] Over 10,000 kroons a month.
  \end{inparaitem}
  2008 and later, the coding is:
  \begin{inparaitem}
  \item[2)] Vähemalt 1000 krooni kuus;
  \item[3)] Vähemalt 2000 krooni kuus;
  \item[4)] Vähemalt 3000 krooni kuus;
  \item[5)] Vähemalt 4000 krooni kuus;
  \item[6)] Vähemalt 5000 krooni kuus;
  \item[7)] Vähemalt 10 000 krooni kuus;
  \item[8)] Vähemalt 15 000 krooni kuus;
  \item[9)] Vähemalt 20 000 krooni kuus;
  \item[10)] Üle 20 000 krooni kuus.
  \end{inparaitem}
\item[searchParttime] Searching full-time or part-time job:
  \begin{inparaitem}
  \item[1] only full-time;
  \item[2] full-time, would agree with part-time;
  \item[3] part-time, would agree with full-time;
  \item[4] only part-time
  \end{inparaitem}
\item[prevIndustry] previous industry for the unemployed and
  inactive.  Codes as for \emph{industry}:
  \begin{inparaitem}
  \item[A] Agriculture, hunting and forestry;
  \item[B] Fishing;
  \item[C] Mining and quarrying;
  \item[D] Manufacturing;
  \item[E] Electricity, gas and water supply;
  \item[F] Construction;
  \item[G] Wholesale and retail trade; repair of motor vehicles etc.;
  \item[H] Hotels and restaurants;
  \item[I] Transport, storage and communication;
  \item[J] Financial intermediation;
  \item[K] Real estate, renting and business activities;
  \item[L] Public administration and defence; compulsory social security;
  \item[M] Education;
  \item[N] Health and social work;
  \item[O] Other community, social and personal service activities;
  \item[P] Activities of households;
  \item[Q] Extra-territorial organizations and bodies.
  \end{inparaitem}
\item[prevOccupation] Previous occupation for the unemployed and
  inactive.  Except for 1998, coded as \emph{occupation}:
  \begin{inparaitem}
  \item[0] Armed forces;
  \item[1] Legislators, senior officials and managers;
  \item[2] Professionals;
  \item[3] Technicians and associate professionals;
  \item[4] Clerks;
  \item[5] Service workers and shop and market sales workers;
  \item[6] Skilled agricultural and fishery workers;
  \item[7] Craft and related trade workers;
  \item[8] Plant and machine operators and assemblers;
  \item[9] Elementary occupations;
  \end{inparaitem}

  For 1998, the codes are 1, 2 and 3, probably meaning primary,
  secondary, and tertiary sector.
\end{description}


\section{Further Ideas}

\selectlanguage{estonian}
\begin{description}
\item[tööturu ajalugu] Kas inimene on olnud viimase aasta (3 aasta)
  jooksul töötu?  Mitteaktiivne?  Kas see on tema esimene töökoht?
  Tuleks uurida mis on tehtav.

\item[tervis] seda mida ETUst saab

\item[kaalud] ETU on mingis osas \emph{choice-based} valim ja
  tegelikult tuleks sääl kaaludega arvestada.
\end{description}

\selectlanguage{english}

\section{TODO}
\label{sec:todo}

\begin{itemize}
\item observation id-s seem to be NA-s since about 2000
\end{itemize}

\end{document}
